% !TEX TS-program = pdflatex
% !TEX encoding = UTF-8 Unicode

% This is a simple template for a LaTeX document using the "article" class.
% See "book", "report", "letter" for other types of document.

\documentclass[10pt]{article} % use larger type; default would be 10pt

\usepackage[utf8]{inputenc} % set input encoding (not needed with XeLaTeX)

%%% Examples of Article customizations
% These packages are optional, depending whether you want the features they provide.
% See the LaTeX Companion or other references for full information.

%%% PAGE DIMENSIONS
\usepackage{geometry} % to change the page dimensions
\geometry{a4paper} % or letterpaper (US) or a5paper or....
% \geometry{margin=2in} % for example, change the margins to 2 inches all round
% \geometry{landscape} % set up the page for landscape
%   read geometry.pdf for detailed page layout information

\usepackage{graphicx} % support the \includegraphics command and options

% \usepackage[parfill]{parskip} % Activate to begin paragraphs with an empty line rather than an indent

%%% PACKAGES
\usepackage{booktabs} % for much better looking tables
\usepackage{array} % for better arrays (eg matrices) in maths
\usepackage{paralist} % very flexible & customisable lists (eg. enumerate/itemize, etc.)
\usepackage{verbatim} % adds environment for commenting out blocks of text & for better verbatim
\usepackage{listings}
\usepackage{subfig} % make it possible to include more than one captioned figure/table in a single float
\usepackage{multicol}
% These packages are all incorporated in the memoir class to one degree or another...

%%% HEADERS & FOOTERS
\usepackage{fancyhdr} % This should be set AFTER setting up the page geometry
\pagestyle{fancy} % options: empty , plain , fancy
\renewcommand{\headrulewidth}{0pt} % customise the layout...
\lhead{}\chead{}\rhead{}
\lfoot{}\cfoot{\thepage}\rfoot{}

%%% SECTION TITLE APPEARANCE
\usepackage{sectsty}
\allsectionsfont{\sffamily\mdseries\upshape} % (See the fntguide.pdf for font help)
% (This matches ConTeXt defaults)

%%% ToC (table of contents) APPEARANCE
\usepackage[nottoc,notlof,notlot]{tocbibind} % Put the bibliography in the ToC
\usepackage[titles,subfigure]{tocloft} % Alter the style of the Table of Contents
\renewcommand{\cftsecfont}{\rmfamily\mdseries\upshape}
\renewcommand{\cftsecpagefont}{\rmfamily\mdseries\upshape} % No bold!

\usepackage{pifont}
\usepackage{color}
\usepackage[dvipsnames]{xcolor}
\definecolor{lightgray}{gray}{0.85}

%%% END Article customizations

%%% The "real" document content comes below...

\title{Whiley Comparison}
\author{Vivian Stewart}
%\date{} % Activate to display a given date or no date (if empty),
         % otherwise the current date is printed 

\begin{document}
\maketitle

\lstloadlanguages{Java}
\lstset{
        language=Java,
        keywords={function, method, type, constant, assert, assume, for, while, switch, is, if, case, return,
          else, package, define, as, requires, ensures, where, no, new,
          all, bool, int, byte, char, string, void, real, in, any,
          null, var, public, protected, private, native, export, skip, break, do,
          throws, catch, continue, default, try
        },
        basicstyle=\ttfamily\tiny,
        commentstyle=\rmfamily\tiny,
        stringstyle=\itshape\tiny,
        moredelim=*[s][commentstyle]{/*}{*/}, % allows keyword highlighting inside comments
        morecomment=[l][commentstyle]{//},      % single line comments are set by...
        backgroundcolor=\color{lightgray},
	 commentstyle=\color{Sepia},
	 keywordstyle=\color{MidnightBlue},
        frame=single, % adds a frame around the code
%        frameround=tttt,
        framesep=0.25cm,
        texcl,                                                          % ...LaTeX
        moredelim=[is][\ttfamily]{<code>}{</code>}, % allows setting code inside multi-line comments
        moredelim=*[s][\ttfamily]{/*@}{*/}, % JML annotations
        moredelim=*[l][\ttfamily]{//@}, % JML annotations
        moredelim=**[is][\itshape]{/_}{_/}, % allows emphasizing sub-expressions
        moredelim=**[is][\bfseries]{/b_}{_b/}, % allows emphasizing sub-expressions
        escapeinside={(*@}{@*)}, %use (*@\label{line:desc}@*) to label lines for \ref
        mathescape=true,                % allow $ $ for math mode (will this break things?}
        tabsize=4,
        tab=\rightarrowfill,
        xleftmargin=0.25cm,
        xrightmargin=0.25cm,
        showspaces=false,
        showtabs=false,
        columns=fullflexible,
        numberstyle=\tiny,
        numbers=left,
        keepspaces=true,
        mathescape=true, % allows $ to switch in and out of math mode within listings
        literate={tau}{{$\tau$}}1 {tau'}{{$\tau\prime$}}1 {(|}{{$\lpbar$}}1 {---}{{$\hole$}}1
           {-*-}{{$\times$}}1 {||_}{{$\lceilfloor$}}1 {_||}{{$\rceilfloor$}}1
           {/0}{{$\emptyset$}}1 {/bul}{{$\unit$}}1 {:->}{{$\mapsto$}}1
           {~}{$\sim$}1,
}

\begin{abstract}

\end{abstract}

\section{Introduction}
From an undergraduate perspective one might approach verification with optimistic positivity, ``I can go about proving all of the programs I've contributed to, all of the weird algorithms I've always wondered about...'' However the truth of Software Verification is in actuality much more of a field of research with very limited tools and a steep learning curve. Current efforts are impressive from an academic point of view and but require years of invested effort to learn the various analyses and how these fit together to prove correctness of program execution.

There are some interesting tools that have been developed recently, making for an easier more tractable learning curve and easing the burden of proof for the programmers whom may be a bit intimidated by formal specification or those who found other tools too much effort for meagre results.

As programmers we need some tools to assure us that, the code we work so long and hard on, does what we think it does. There are linter's, Unit testing, Static Analysis tools, all of which will give you some idea about some aspect of code execution but Formal Verification promises to cover much greater ground with relative ease. First Order Logic is expressed alongside code to describe program execution which is then converted into verification conditions to be checked and asserted by an automated proof assistant. It is possible to completely or partially specify a program depending on critical importance correctness and impact of failure.

Formal Logic is at first difficult to grasp for the majority of programmers and even the tools discussed here in require some background in Mathematical Logic, proof by contradiction, contrapositive and induction etc. Induction seems to be a tricky one not only for human beings but for automated deduction also, for instance, Z3 doesn't support induction (recursive) verification conditions, though there are structures available in Dafny for inductive proof, lemmas, functions, predicates (possibly in boogie?) One interesting thin that caught me out was the order of statements in the specification was important which I thought unusual as one would tend to think of certain  mathematical operations as commutative or associative and mathematical definitions as being declarative. But it would seem that logical operations in specifications employ short circuit logic which makes order of premises important.

More often than not code made for one tool is easily transcribed to the other as long as there is a one to one mapping of structures used in formal verification of the program. The difference in range notation between the different tools is note worthy as in Dafny ranges can simply be seen mathematically \verb/0 <= x < length/, where as Whiley is \verb/x in 0..length/ and Spec\# \verb/x in (0:length)/ or \verb/x in (0..length - 1)/. In Whiley the \verb/a..b/ means mathematically $[a,b)$. In Spec\# \verb/(a:b)/ means  $[a,b)$ but \verb/(a..b)/ means $[a,b]$. I personally prefer Dafny's way for it's clarity though the others are short and concise, but this is a matter of taste. Potentially confusing off by one errors can be hard to see.

Intuitively one has to establish a correspondence between any of the variables that change in the loop(s) of the program. Knowing if there has been enough specification or even a complete specification is not immediate and it is easy to come up with only a partial specification. Hence, it raises two main questions: how to be sure that our specification is complete, and how complete must a specification be. For the first question, one must reason on some model of the specified program. For the second question, no definitive answer is known. It depends on the context in which the specification is written and the kind of properties that must be established. The amount of specification required for a given function is very different when verifying a given property for a given application in which calls to the function always occur in a well-defined context and when specifying it as a library function which should be callable in as many contexts as possible.

Overflow, correspondence between bounded and unbounded types inside a specification...

Termination...

Null and non-null types...

Simplification and it's effects time taken to verify and translation into more/less verification conditions.

Embedded code in specifications... lemmas, Pure Functions, Predicates...

\section{Benchmarks}
Selected benchmarks were intended to be simple since the verification of large blocks of code are exponentially asymptotic in logical deduction and\slash or not decidable in the worst case. The benchmarks range from simple to reasonably complex. Some of them were exercises recommended by David Pearce, where as others were suggested by Lindsey Grooves. To begin with, the main Idea of this project was to create some small programs for use in lecture slides or as examples in tutorials. 

\subsection{palindrome} 
Scans through an array from the outer elements inwards to the centre, searching for a mismatch in values in which case returning false if there is such a mismatch and returning true if there is no such mismatch.

\subsection{firstIndexOf}
Finds the first index where the specified element is found in the given array of integers, searching through the array from the zero index in the positive direction. returns immediately the item is found, otherwise traverses the entire array then returns -1 to represent the absence of the value in the array.

\subsection{lastIndexOf}
Finds the last index where the specified element is found in the given array of integers, searching through the array from the the highest index in the negative direction. A lot like the firstIndexOf function in that the entire array is traversed if the value is not found but quits early if the value is found.

\subsection{max}
Searches the entire list in order to find and return the largest integer found. Accumulating the maximum value evaluated so far, as the algorithm traverses the array. 

\subsection{occurences}
Returns the number of times a given value is encountered while traversing the array.

\subsection{strlen}
The \textbf{strlen} function is intended to mirror the C function of the same name. It counts the number of characters in a string (array) of characters, having come to the end of the list when encountering a null character. The null character is usually denoted as '\textbackslash0', but since there is no char type currently available in Whiley, instead we have to use an abstraction of this data type. Ascii characters are bytes (8 bits) their range of values is between 0 and 255 inclusive and the null character is 0.

\subsection{linearSearch}
Much like the firstIndexOf function but iterates through a sorted list and can therefore quit early when the element matching the value is either found or the value is exceeded by the current element. Previous versions found an index of insertion into the sorted part of the array.

\subsection{binarySearch}
Using the divide and conquer method to find the index of an element in the given array matching the given value.

\subsection{append}
Adds a single element to the end of the array by making a new array that is one element larger than the original and copies the original array in order to the lower elements and inserting the given value in the last index.

\subsection{remove}
The remove function should remove the item at the given index from the given array of integers and return the resulting array otherwise unchanged. The resulting array is of course one element shorter in length. This is done by creating a new array one smaller in size and then copying across the elements before the given index directly, followed by copying across elements above the index to a position one index lower and overwriting the removed element.

\subsection{copy}
Copies a region of the source array into a same sized region of the destination. 

\subsection{displace}
rotates a region of the array by one place forward

\subsection{insert}
This function should insert the item at the given index from the items array.  The resulting array is of course one element longer in length.

\subsection{insertionSort}
...
\section{Results \& Discussion}

\begin{center}
\begin{tabular}{| l | c | c | c | c | c | c | c |}
	\hline
	Env. & Palin. & FIO & LIO & Max & Occ & Slen & LSrch \\ \hline \hline
	Whiley & \ding{51} & \ding{51} & \ding{51} &\ding{51} & \ding{56} & \ding{51} & \ding{51} \\ \hline
	Dafny & \ding{51} & \ding{51} & \ding{51} &\ding{51} & \ding{51} & \ding{51} & \ding{51} \\
	\hline
\end{tabular}
\end{center}

\begin{center}
\begin{tabular}{| l | c | c | c | c | c | c | c |}
	\hline
	Env. & BSrch & Appd & Remv & Copy & Dplc & Inst & InSrt \\ \hline \hline
	Whiley & \ding{56} & \ding{51} & \ding{56} &\ding{56} & \ding{56} & \ding{56} & \ding{56} \\ \hline
	Dafny & \ding{51} & \ding{51} & \ding{51} &\ding{51} & \ding{51} & \ding{51} & \ding{51} \\
	\hline
\end{tabular}
\end{center}
the loop invariants tended to be reflected in the postconditions.

Z3 has no inductive proof mechanism and I suspect this is why Dafny has such a messy inductive definition syntax that I also suspect derives from boogie rather than Z3.

Whiley allows you to do more with less, minimal syntax but more expressive. (IMPORTANT!! flexible language constructs)… 

Dafny has more complicated syntax to deal with memory management i.e pointers\slash references, lots to remember and coordinate.

The tools Dafny provides are for the development of proofs in a “top-down” manner, and which allow us to concentrate on the “architecture” of the proof. Unlike Frama-C.

With Frama-C in particular, it is a 2-valued logic with only total functions. Consequently, expressions are never “undefined”. Having only total functions implies that one can write terms such as 1\slash 0, or *p when p is null.


Specifications in Frama-C can have implicit casts between C-types and Mathematical types (something to watch out for.)
Weird:

\begin{itemize}
\item 5\slash 3 is 1 and 5 \% 3 is 2;
\item (-5)\slash 3 is -1 and (-5) \% 3 is -2;
\item 5\slash (-3) is -1 and 5 \% (-3) is 2;
\item (-5)\slash (-3) is 1 and (-5) \% (-3) is -2.
\end{itemize}

Frama-C When no 'assigns' clauses are specified, the function is allowed to modify every visible variable.
\begin{description}
\item[Termintes]It is possible to relax a particular function’s specification by providing a formula that describes the conditions in which the function is guaranteed to terminate. 
\item[Assertions]when the analyzer is not able to determine that an assertion always holds, it may be able to produce a pre-condition for the function that would, if it was added to the function’s contract, ensure that the assertion was verified. 
\item[Overflow] overflow is\slash must be handled.
\end{description}

\lstset{language=C,morekeywords={
	case,inductive,behavior,behaviors,requires,ensures,assumes,complete,disjoint,\old,\forall,\valid
}}

\begin{lstlisting}
	inductive reachable{L} (list* root, list* node) {
	  case root_reachable{L}:
	    \forall list* root; reachable(root,root);
	  case next_reachable{L}: // L indicates a Label -> memory state
	    \forall list* root, *node;
	      \valid(root) ==> reachable(root -> next, node) ==>
	        reachable(root,node);
\end{lstlisting}
\begin{lstlisting}
	requires \valid(p) && \valid(q);
	ensures *p <= *q; // pointers
	behavior p_minimum:
		assumes *p < *q;
		ensures *p == \old(*p) && *q == \old(*q);
	behavior q_minimum:
		assumes *p >= *q;
		ensures *p == \old(*q) && *q == \old(*p);
	complete behaviors p_minimum, q_minimum;
	disjoint behaviors p_minimum, q_minimum;
\end{lstlisting}

No verifier can tell you whether your code doesn't work the according to the specification or the specification doesn't describe what the code does, as this depends on the intention of the user\slash developer.
Frama-C ACSL specification language lacks flexible language constructs, very cumbersome\slash specialised syntax.

All languages need different forms of verification and to different degrees. Sensitivity to approach to the problem.
\subsection{Whiley}
	Whiley is very easy to use and shows promise as a tool to verify programs in such a way that the accompanying specification scales with complexity of the program to be verified. Whiley offers a simplified program and specification definition. However, Whiley's underlying prover needs some work to bring Whiley up to the level attained by other tools. Arrays in Whiley are not reference or pointer types, there is no possibility of the array being null, also copying of an array is just a matter of assigning that array to a different variable, bypassing the issue of specifying array copying loops. Some basic arithmetic and region ranges seem to be difficult for the whiley constraint solver (WyCS.)
\subsection{Dafny}
	Dafny is a sophisticated tool\slash language for verification based on Microsofts' Z3 prover and Boogie with Monodevelop\slash Visual Studio. Though tests were performed on the command line to gain feedback from Dafny. Dafny can use pure functional code to help prove the correctness of imperative code for instance, only functions can used in specification unless qualified as 'function method'. These pure function definitions require very minimal specification in order to provide lemmas, axioms and other mathematical constructs to prove correct execution. One thing that struck me as odd was the use of short circuit logic in the specifications which implies a certain order to the specifications statements. Dafny doesn't require return statements in the usual way and defines a return variable(s) that is assigned to, and if necessary a 'return' can be called for early return from the call. Explicit and obvious reference to the returned variable was very convenient. All function\slash method parameters are immutable unless specified otherwise.
	Inside the specifications arrays are treated differently from other containers but are implicitly convertible to the built in sequence type if needed. Which allows for the very convenient syntax: arr1[m..n] == arr2[m..n] for an adjacent element comparison avoiding the need for long winded predicates. But not so convenient in that a sequence cannot be easily converted back to an array. Arrays are always possibly null in Dafny and the there are attempts in the specification language to mitigate the issue of reference aliasing and other such memory management quirks (modifies clause.) Unlike Whiley arrays are of reference type and need to be null checked.
\subsection{Frama-C}
	When it comes to verifying C code Frama-C seems to be a current industry favourite. But I found the user gtk interface to be cumbersome and confusing. The ACSL language would appear to appeal more to an expert C programmer which I am not. Much expertise is needed to master Frama-C and every aspect and plugin have large manuals and tutorials available, all very heavy in detail. What I really wanted was a more general tool that would help point out where errors in my reasoning were evident. Frama-C offers only the vaguest of clues and errors in my reasoning are not always obvious. The issues of bounded types are dealt with up front and checked with the WP plugin through an RTE (runtime error) option that injects (over\slash under)flow checks. Specifications are meant to be written in a style that complements C and is analogous to C, even mirroring C conventions and C arithmetic. The lack of consolidation makes the various plugins difficult to integrate into a workflow. The ACSL specification language is littered with predefined predicates and detailed set of proof tools such as user defined predicates, behaviours, inductive(recursive) functions and axioms etc. Arrays were a very sticky issue in Frama-C and involve proof of non-null and the validation of the range of indices and elements. For loop invariants, proof of termination is provided by the loop 'variant' key word, which is hard to see in amongst loop 'invariant' clauses in the same comment code block. Also there are assigns clauses that define the frame of modification for both the function and loop invariants (loop assigns.) since unlike other tools Frama-C has no knowledge of what has been modified in the function or inside the loop (side effects). Since the C language does not have formal semantics one should take the verification of C code with a grain of salt, horribly coded nonsense with undefined results can still be verified.
\subsection{Spec\#}
	Spec\# having the same underlying workings as Dafny (a.k.a Boogie and Z3) if there are many differences between the two.
\newpage
\section{Appendix}
\begin{multicols}{2}
\lstinputlisting{Whiley/01_palindrome.01.whiley}
\lstinputlisting{Whiley/02_firstIndexOf.01.whiley}
\lstinputlisting{Whiley/03_lastIndexOf.01.whiley}
\columnbreak
\lstinputlisting{Dafny/01_palindrome.01.dfy}
\lstinputlisting{Dafny/02_firstIndexOf.01.dfy}
\lstinputlisting{Dafny/03_lastIndexOf.01.dfy}
\newpage
\lstinputlisting{Whiley/04_max.01.whiley}
\lstinputlisting{Whiley/05_occurences.01.whiley}
\lstinputlisting{Whiley/06_strlen.01.whiley}
\columnbreak
\lstinputlisting{Dafny/04_max.01.dfy}
\lstinputlisting{Dafny/05_occurences.01.dfy}
\lstinputlisting{Dafny/06_strlen.01.dfy}
\newpage
\lstinputlisting{Whiley/07_linearSearch.01.whiley}
\lstinputlisting{Whiley/08_binarySearch.01.whiley}
\lstinputlisting{Whiley/09_append.01.whiley}
\columnbreak
\lstinputlisting{Dafny/07_linearSearch.01.dfy}
\lstinputlisting{Dafny/08_binarySearch.01.dfy}
\lstinputlisting{Dafny/09_append.01.dfy}
\newpage
\lstinputlisting{Whiley/10_remove.01.whiley}
\lstinputlisting{Whiley/11_copy.01.whiley}
\columnbreak
\lstinputlisting{Dafny/10_remove.01.dfy}
\lstinputlisting{Dafny/11_copy.01.dfy}
\newpage
\lstinputlisting{Whiley/12_displace.01.whiley}
\columnbreak
\lstinputlisting{Dafny/12_displace.01.dfy}
\newpage
\lstinputlisting{Whiley/13_insert.01.whiley}
\columnbreak
\lstinputlisting{Dafny/13_insert.01.dfy}
\end{multicols}
\end{document}
